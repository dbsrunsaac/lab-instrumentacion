%============================================================================
%      PLANTILLA LABORATORIO DE INSTRUMENTACÍON ELECTRÓNICA 
%               INGENIERÍA ELECTRONICA - UNSAAC
%============================================================================

\documentclass[a4paper,12pt]{book}

\renewcommand{\rmdefault}{phv}  % Cambiar la fuente a Helvetica (similar a Arial)
\renewcommand{\sfdefault}{phv}  % Cambiar la fuente sans-serif a Helvetica

\usepackage{./estilos/estiloBase}
\usepackage{./estilos/colores}  
\usepackage{./estilos/comandos} 



\addbibresource{bib.bib}


%\input{Anexo-Siglas.tex}


%%%% PARA LA NOMENCLATURA
\renewcommand{\nomgroup}[1]{%
\ifthenelse{\equal{#1}{A}}{\item[\textbf{T\'erminos Comunes}]}{%
\ifthenelse{\equal{#1}{B}}{\item[\textbf{Caso: URLLC y eMBB}]}{%\\     SE CAMBIA "ESCENARIO" POR "CASO"
\ifthenelse{\equal{#1}{C}}{\item[\textbf{Caso: URLLC y mMTC}]}{%
\ifthenelse{\equal{#1}{D}}{\item[\textbf{Escenario 3}]}{}}}}
}


\begin{document}



\setlength{\parskip}{4mm}	% Separaci??n de los p??rrafos
\overfullrule=2cm			% Muestra una l??nea negra en el borde de la
							% p??gina cuando se superen los m??rgentes
\sloppy						% permite espacios m??s grandes de lo normal
							% para cuadrar las l??neas

%====================== CITAS ============================
\hypersetup{				%Requerido para configurar el comportamiento de los enlaces en el documento
     colorlinks=true,		%Color enlaces (es decir, no más cuadros alrededor de los enlaces)
     linkcolor=blue,		%los enlaces a los archivos locales se establecen en color xxxxxx
     filecolor= yellow,	
     citecolor = blue,  	%color de enlace de citas
     linkbordercolor=white,
     urlcolor=blue,
     pdfborderstyle={/S/U/W 1}}
%====================== CITAS ============================

%\pagestyle{empty}
\begin{titlepage}

  \begin{center}
    \Large{\textbf{UNIVERSIDAD NACIONAL SAN ANTONIO ABAD DEL CUSCO}} \\ 
    \vspace{0.4cm}
    \large{FACULTAD DE INGENIER\'IA EL\'ECTRICA, ELECTR\'ONICA, INFORM\'ATICA Y MEC\'ANICA}\ \\ 
%    \vspace{0.1cm} 
%    \Large{DEPARTAMENTO ACAD\'EMICO DE INGENIER\'IA ELECTR\'ONICA}\ \\ 
    \vspace{0.1cm} 
    \large{ESCUELA PROFESIONAL DE INGENIER\'IA ELECTR\'ONICA}\ \\ 
	
    \vspace{1.0cm}    
    
%    \includegraphics[width=0.2\textwidth]{images/UNSAAC.png}\\% \hspace{0.1mm} 
    \includegraphics[width=0.4\textwidth]{images/LI-UNSAAC.png} \\


    \vspace{0.8cm}

    \large{\textbf{\textsc{LABORATORIO DE INSTRUMENTAC\'ION ELECTR\'ONICA}}} \\
    \vspace{0.2cm}
    \large{\textbf{\textsc{INFORME PREVIO N° 1:}}} \\ 
    
    
    \vspace{0.5cm}
    \large{ \textsc{<<TENSI\'ON DE OFFSET EN AMPLIFICADORES OPERACIONALES>>}} \\
    \vspace{1.0cm}
    
    \begin{table}[H]
	\centering
	\begin{tabular}{rl}
	\large{\textbf{Autor:}}   & \large{Davis Bremdow Salazar Roa}  \\
        \large{\textbf{Codigo:}}   & \large{200353}  \\
        \large{\textbf{Docente:}} & \large{Ing. Jose Luis Flores Vasquez}
	\end{tabular}
	\end{table}

	
	\vspace{0.2cm}

    \Large{Cusco -- Per\'u \\
%    \vspace{0.2cm}
    Septiembre, 2025}
    
  \end{center}
\end{titlepage}

%\cleardoublepage

%\chapter*{Agradecimientos}
%\input{0.1-Agradecimientos}
%\cleardoublepage

%\frontmatter % Introducci??n, ??ndices ...
%\pagestyle{plain}
%\input{0.2-Abstract.tex}
%\cleardoublepage

%\tableofcontents
%\cleardoublepage
%\phantomsection

%\addcontentsline{toc}{chapter}{Resumen}
%\addcontentsline{toc}{chapter}{Abstract}
%\addcontentsline{toc}{chapter}{Resum}

%\listoftables
%\listoffigures

%\mainmatter % Contenido en si ...

%\chapter{Introducci\'on}\label{cap:Introduccion}




\section*{Introducción}

\section*{Resultados finales}










%%% ESTILO APA %%%%%
%--------------------

\printbibliography
%\bibliographystyle{apalike}
%\bibliography{bib.bib}
%\printbibliography

\printglossary[type=\acronymtype, title={Lista de Abreviaturas}, toctitle={Lista de Abreviaturas}]
%\printglossary[type=\acronymtype, title=T\'erminos y Abreviaturas]




\end{document}
